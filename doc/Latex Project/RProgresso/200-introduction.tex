\setcounter{secnumdepth}{2}
\chapter{Introduction}
\label{introduction}
\thispagestyle{plain}

Many people believe that AI (Artificial Intelligence) was created to imitate human behavior and the way humans think and act. Even though people are not wrong, AI was also created to solve problems that humans are unable to solve, or to solve them in a shorter time window, with a better solution. Humans may take days to find a solution, or may not find a solution that fits their needs. Optimization algorithms, that is, methods that seek to minimize or to maximize some criterion, may deliver a very good solution in minutes, hours or days, depending on how much time the human is willing to use in order to get a proper solution.\\
\\
A concrete example of this type of problems is the creation of timetables. Timetables can be used for educational purposes, sports scheduling, transportation timetabling, among other applications. The timetabling problem consists in scheduling a set of events (e.g., exams, people, trains) to a specified set of time slots, while respecting a predefined set of rules constraints. In some cases the search space is so limited by the constraints that one is forced to relax them in order to find a solution. 


\section{Timetabling Problems}

Timetabling problems consists in scheduling classes, lectures or exams on a school or university depending on the timetabling type. These types of timetabling may include scheduling classes, lectures, exams, teachers and rooms in a predefined set of timeslots which are scheduled considering a set of rules. These rules can be, for instance, a student can't be present in two classes at the same time, a student can't have two exams on the same day or even an exam must be scheduled before another.\\
\\
Timetabling is divided into various types, depending on the institution type and if we're scheduling classes/lectures or exams. Timetabling is divided in three main types:

\begin{itemize}
	\item Examination timetabling - consists on scheduling university exams depending on different courses, avoiding the overlap of exams containing students from the same course and spreading the exams as much as possible in the timetable;
	\item Course timetabling - consists on scheduling lectures considering the multiple university courses, avoiding the overlap of lectures with common students;
	\item School timetabling - consists on scheduling all classes in a school, avoiding the need of students being present at multiple classes at the same time.
\end{itemize}

In this project, the main focus is the examination timetabling problem. \\

The process of creating a timetable requires that the final solution follows a set of constraints. These sets differ depending on the timetabling type and problem specifications. This constraints are divided in two groups: \textit{hard constraints} and \textit{soft constraints}. Hard constraints are a set of rules which must be followed in order to get a working solution. On the other hand, soft constraints represent the views of the different interested players (e.g., institution, students, nurses, train operators) in the resulting timetable. The satisfaction of this type of constraints is not mandatory as is for the case of the hard constraints. In the timetabling problem, the goal is usually to optimize a function with a weighted combination of the different soft constraints, while satisfying the set of hard constraints. 

\section{Objectives}
\label{section:Objts}

This project's main objective is the production of a prototype application which serves as a examination timetabling generator tool. The problem at hand focus on the specifications submitted on the International Timetabling Competition 2007 (ITC 2007), \textit{First track}, which includes 12 benchmark instances. In ITC 2007 specifications, the examination timetabling problem considers a set of periods, room assignment, and the existence of constraints considering real problems.\\
\\
The application's features are as follows:

\begin{itemize}
	\item Automated generation of examination timetabling, considering the ITC 2007 specifications (\textit{mandatory});
	\item Validation (correction and quality) of a timetable provided by the user (\textit{mandatory});
	\item Graphical User Interface to allow the user to edit generated solutions and to optimize user's edited solutions (\textit{optional}).
\end{itemize}

This project is divided in two main phases. The first phase consists on studying some techniques and solutions for this problem emphasizing meta-heuristics like: \gls{ga} , \gls{sa}, \gls{ts}, and some of its hybridizations. The second phase is based on developing, rating and comparing solutions, using ITC 2007 data.

%The goal of the present project is to create an examination timetable generator using the ITC 2007 specification. The solutions will be validated using a validator developed for this purpose in order to assess the quality of the solution. It is also required that the developed prototype may work with two exam epochs which can be considered an extension to the ITC-2007 formulation. In the end an (optional) Graphical User Interface will be created in order to allow the user to edit the current solution to fit the users needs and and to allow the optimization of the edited solution. The generator will be tested using data from ITC-2007 and some actual data from six different programs lectured at ISEL.

\section{Planning}

The main objective of the first part of this work is to check on the state-of-the-art considering various problems and solutions.
\\
The second part's main goal is to develop solutions to the ITC 2007 problem.  One of the approaches to be developed will use a local search algorithm (e.g., Simulated Annealing) to improve the solution obtained using a graph coloring heuristic. The results of this approach will be compared against other approaches applied to ITC 2007 timetabling problem, namely the first five winners. The second approach, instead of using a single-solution based meta-heuristic, it uses a population-based meta-heuristic (e.g., Genetic Algorithm) with a local search algorithm (e.g., Simulated Annealing), or a combination of other meta-heuristics. In the end, after comparing results and taking conclusions, a GUI to edit generated solutions may be created, as an optional feature.\\
\\
The GANTT diagram presented on Figure \ref{fig:planning} shows a detailed planning for the development of all the main topics of this project including both phases mentioned above. The colors presented on the diagram are green and blue, which designate the first and second phase, respectively.\\

\begin{figure}[t!]
\centering
\begin{ganttchart}[%Specs
     y unit title=0.5cm,
     y unit chart=1.0cm,
     vgrid,hgrid,
     bar label font=\scriptsize{}, %\tiny
     bar label node/.append style=%
		{align=left},
     title height=1
    ]{1}{26}
    %labels
    \gantttitle[
    ]{2014}{8}
    \gantttitle[]{2015}{18} 
    \\              
    \definecolor{foobarblue}{RGB}{0,153,255}
    \definecolor{foobargreen}{RGB}{50,200,50}
    \gantttitle{09}{2}
    \gantttitle{10}{2}
    \gantttitle{11}{2}
    \gantttitle{12}{2}
    \gantttitle{01}{2}
    \gantttitle{02}{2}
    \gantttitle{03}{2}
    \gantttitle{04}{2}
    \gantttitle{05}{2}
    \gantttitle{06}{2}
    \gantttitle{07}{2}
    \gantttitle{08}{2}
    \gantttitle{09}{2}
    \\
    \ganttbar[bar/.append style={fill=foobargreen!60}]
    	{Report\\development}{6}{7}
    \ganttbar[bar/.append style={fill=foobargreen!60}]{}{12}{14}
    \ganttbar[bar/.append style={fill=foobarblue!60}]{}{15}{26}
    \\
    \ganttbar[bar/.append style={fill=foobargreen!60}]
    	{Discussion of the\\ proposed idea}{5}{6}
   	\ganttbar[bar/.append style={fill=foobargreen!60}]{}{14}{14}
    \\
    \ganttbar[bar/.append style={fill=foobargreen!60}]
    	{Study of \\ heuristic \\algorithms}{5}{7}
    \ganttbar[bar/.append style={fill=foobargreen!60}]{}{12}{13}
    \\
    \ganttbar[bar/.append style={fill=foobargreen!60}]
    	{Study of the \\ state of art}{6}{7}
    \ganttbar[bar/.append style={fill=foobargreen!60}]{}{12}{14}
    \ganttbar[bar/.append style={fill=foobarblue!60}]{}{15}{26}
    \\
    \ganttbar[bar/.append style={fill=foobargreen!60}]
    	{Progress report \\development}{13}{14}
    \\
    \ganttbar[bar/.append style={fill=foobargreen!60}]
    	{Progress report \\ revision and \\ submission}{14}{14}
    \\
    \ganttbar[bar/.append style={fill=foobargreen!60}]
    	{Definition of \\ project's \\ architecture}{15}{15}
     \ganttbar[bar/.append style={fill=foobarblue!60}]{}{16}{16}
     \ganttbar[bar/.append style={fill=foobarblue!60}]{}{19}{19}
    \\
    \ganttbar[bar/.append style={fill=foobarblue!60}]
    	{Single solution \\ development}{15}{16}
    \\
    \ganttbar[bar/.append style={fill=foobarblue!60}]
    	{Validator \\ implementation}{17}{18}
    \\
    \ganttbar[bar/.append style={fill=foobarblue!60}]
    	{Single solution \\ results comparison}{18}{19}
    \\
    \ganttbar[bar/.append style={fill=foobarblue!60}]
    	{Populated solution \\ development}{20}{21}
    \\
    \ganttbar[bar/.append style={fill=foobarblue!60}]
    	{Populated solution \\ results comparison}{21}{22}
    \\
    \ganttbar[bar/.append style={fill=foobarblue!60}]
    	{Second progress \\ report development \\ and submission}{22}{22}
    \\
    \ganttbar[bar/.append style={fill=foobarblue!60}]
    	{Work demonstration}{22}{22}
    \\
    \ganttbar[bar/.append style={fill=foobarblue!60}]
    	{Drawing \\ conclusions}{22}{24}
    \\
    \ganttbar[bar/.append style={fill=foobarblue!60}]
    	{Thesis revision \\ and submission}{22}{26}

\end{ganttchart}

\caption{GANTT diagram for project planning}
   \label{fig:planning}
\end{figure}

%The planning shown on the GANTT diagram may not be accurate. Some topics and features may take less or more time than specified. The developing order of certain features may change depending on the necessity of these to implement others already in development.

\section{Document Organization}

In section \ref{stateofart} the State-of-the-Art is introduced specifying the timetabling problem focusing on examination timetabling and existing approaches including ITC 2007's problem. In the existing approaches, some of the most used heuristics are mentioned and briefly explained. In section 4 we explain the implementation, organization and modeling of the solutions, including the loader, which is given by the provided by this thesis' coordinators, and the used heuristic methods for solving ITC 2007's problem. This document ends by explaining the conclusions taken after implementing all the project's features and comparing the results with others approaches present on the state-of-art.