\chapter{State of the Art}
\label{stateofart}
\thispagestyle{plain}

In this section, we review the state of the art of the problem at hand. We start by describing why timetabling is a rather complex problem, some possible approaches on trying to solve it and some of the solutions already taken, specifically for the ITC 2007 benchmarks.\\

\section{Timetabling Problem}

When solving timetabling problems, it is possible to generate one of multiple types of solutions which are \textit{feasible}, \textit{non feasible}, \textit{optimal} or \textit{sub-optimal}. A feasible solution solves all the mandatory problem constraints, in contrary to non feasible solutions. An optimal solution is the best feasible solution possible considering the problem and its optimal solution value. A problem may have multiple optimal solutions. For last, non-optimal solutions are feasible solutions that can't reach the optimal solution value and so are not as good compared to an optimal solutions.\\
\\
Timetabling automation is a subject that has been a target of research for about 50 years. The timetabling problem may be formulated as a search or an optimization problem~\cite{Schaerf1999}. As a search problem, the goal consists on finding a feasible solution that satisfies all the hard constraints, while ignoring the soft constraints. On the contrary, posing the timetabling problem as an optimization problem, one seeks to minimize (considering a minimization problem) the violations of soft constraints while satisfying the hard constraints. Typically, the optimization is done after using a search procedure for finding an initial feasible solution.\\
\\
The basic examination timetabling problem, where only the clash (conflict) hard constraint is observed, reduces to the graph coloring problem~\cite{Jensen2001}. This is a well studied hard problem. Deciding whether a solution exists in the graph coloring problem is a NP-complete problem~\cite{Arora2009}. Considering the graph coloring as an optimization problem, it is proven that the task of finding the optimal solution is a NP-Hard problem \cite{Arora2009}. Graph Coloring problems are explained further in Section \ref{section:ExistingAppr}
\\
%%%%%%%%%%%%%%%%%%%%
\section{Existing Approaches}
\label{section:ExistingAppr}
%%%%%%%%%%%%%%%%%%%%
\begin{figure}[h!]
 \centering
   \includegraphics{./images/typesOfAlgorithms}
   \caption{Types of algorithms adapted from \cite{Talbi2009}.}
   \label{fig:TypesAlgorithms}
\end{figure}

The Figure~\ref{fig:TypesAlgorithms} represents the organization of Optimization methods. These methods are divided into \textit{Exact methods} and \textit{Approximate methods}.\\
\\
Timetabling solution approaches are usually divided in the following categories~\cite{Qu2009}: \textit{exact algorithms} (Branch-and-Bound, Dynamic Programming), \textit{graph based sequential techniques}, \textit{local search based techniques} (Tabu Search, Simulated Annealing, Great Deluge {\color{red} (é?)} ), \textit{population based algorithms} (Evolutionary Algorithms, Memetic algorithms, Ant algorithms, Artificial immune algorithms), \textit{Multi-criteria techniques}, \textit{Hyper-heuristics}, \textit{Decomposition/clustering techniques} and \textit{Hybrid algorithms}, which combine features of several algorithms, comprise the state-of-the-art. Due to its complexity, approaching the examination timetabling problem using exact method approaches can only be done for small size instances. Real problem instances found in practice are usually of large size, making the use of exact methods impracticable. Heuristic solution algorithms have been usually employed to solve this problem.\\
\\
Real problem instances are usually solved by applying algorithms which use both \textit{heuristics} and \textit{meta-heuristics}. Heuristic algorithms are problem-dependent, meaning that these are adapted to a specific problem in which take advantage of its details. Heuristics are used to generate a feasible solution, focusing on solving all hard constraints only. Meta-heuristics on the other hand are problem-independent. These are used to, given the feasible solution obtained using heuristic algorithms, generate a better solution focusing on solving as many soft constraints as possible.\\
\\
Most of the existing Meta-heuristic algorithms belong to one of the three categories: One-Stage algorithms, Two-Stage algorithms and algorithms that allow relaxations~\cite{Lewis2007}. 
\begin{itemize}
  \item The One-Stage algorithm is used to get an initial optimal solution, which the goal is to satisfy both hard and soft constraints at the same time. Approaches using this stage are not very common because it's hard to get proper solutions in a reasonable amount of time trying to simultaneously satisfy both types of constraints;
  \item The Two-Stage algorithms are the most used types of approaches. This category is divided in two phases: the first phase consists in all soft constraints being "discarded" and focusing on solving hard constraints to obtain a feasible solution; the second phase is an attempt to find the best solution, trying to solve the largest number of soft constraints as possible, given the solution of the first phase.
\end{itemize}

\subsection{Exact methods}
\label{subsection:exactmethods}
Approximation algorithms like Heuristics and meta-heuristics proceed to enumerate partially the search space and, for that reason, they can't guarantee finding the optimal solution. On the other side, exact approaches perform a complete enumeration of the search space and thus guarantee that the encountered solution is optimal. A negative aspect is the time taken to find the solution. If the decision problem is very difficult (e.g. NP-Complete), in practical scenarios, if the size of the instances is large, this approach may not be applied due to the long execution time.\\

\subsubsection{Programming Based Technique}
The Constraint Programming Based Technique (CPBT) allows direct programming with constraints which gives ease and flexibility in solving problems like timetabling. Two important features about this technique are backtracking and logical variables that facilitate searching for an optimal solution at the expense of time. Constraint programming is different from other types of programming, as in these types it is specified the steps that need to be executed, but in constraint programming it is specified the properties (hard constraints) of the solution or properties that should not be in the solution~\cite{Qu2009}.\\

\subsubsection{Integer Programming}
The \gls{ip} is a mathematical programming technique in which the optimization problem to be solved must be formulated as an Integer Linear Problem, that is, the objective function and the constraints must be linear, and all problem variables are integer valued. If there are some variables that are continuous and other are integer, then the problem is called Mixed-Integer Linear Programming (MILP). Schaerf~\cite{Schaerf1999} surveys some approaches using the MILP technique to school, course, and examination timetabling.

\subsection{Graph Coloring Based Techniques}
\label{subsection:graphcoloring}
Timetabling problems can be reduced to a graph coloring problem. The usual approaches use graph coloring heuristics in the first phase of the two-stage algorithms. Graph Coloring itself is not an heuristic or meta-heuristic but a method that designates a problem and its variants.\\

\subsubsection{Graph Coloring Problem}
The Graph Coloring (GC) problems consists about assigning colors to an element type of the graph which corresponds to a constraint. The simplest sub-type is the \textit{vertex coloring}, which the main goal is to, given a number of vertices and edges, color the vertices so that no adjacent vertices have the same color. In this algorithm, it's best to find a solution with the lowest number of colors as possible. In examination timetable problem, a basic approach could be to represent the exams as vertices and the hard constraints as edges (considering this is a search algorithm, it is good to use optimization algorithms to deal with soft constraints) so that exams with the same color, can be assign to the same timeslot. After coloring, it proceeds to assign the exams into timeslots considering the colors of the solution~\cite{Qu2009}.\\
\\
Graph Coloring heuristics like \textit{Saturation Degree Ordering} are very commonly used to get the initial solutions. Others like \textit{First Fit} and other \textit{Degree Based Ordering} techniques (\textit{Largest Degree Ordering}, \textit{Incidence Degree Ordering}) are also heuristic techniques for coloring graphs~\cite{Carter1996}.\\

%\paragraph{Saturation Degree Ordering}
%The Saturation Degree Ordering heuristic colors the vertices with more constraints first. The coloring method is as follows: while choosing a vertice to color, the ones with higher saturation degree will be colored first. The saturation degree of one vertice is the number of differently colored vertices adjacent to this vertice or, in another words, the number of different colors of all adjacent vertices. In the case of a tie, the highest saturation vertice with higher number of adjacent vertices is chosen.

\subsection{Meta-heuristics}
Meta-heuristics, as mentioned above, usually provide solutions for optimization problems. In timetabling problems, meta-heuristic algorithms are used to optimize the feasible solutions provided by heuristics, like the GC. Meta-heuristics are divided in two main sub-types, which are \textit{Single-solution meta-heuristics} and \textit{Population-based meta-heuristics}~\cite{Talbi2009}.\\

\subsubsection{Single-solution meta-heuristics}
Single-solution meta-heuristics' main goal is to modify and optimize one single solution, maintaining the search focused in local regions. This type of meta-heuristic is therefore exploitation oriented. Some examples of this type are \textit{\gls{sa}}, \textit{Variable-Neighborhood Search}, \textit{\gls{ts}}, \textit{Guided Local Search}~\cite{Talbi2009}. \\

\subsubsection{Population-based meta-heuristics}
Population-based meta-heuristics' main goal is to modify and optimize multiple candidate solutions, maintaining the search focused in the whole space. This type of meta-heuristic is therefore exploration oriented. Some examples of this type are \textit{Particle Swarm}, \textit{Evolutionary Algorithms}, \textit{Genetic Algorithms}~\cite{Talbi2009}.\\


\subsection{ITC 2007 Examination timetabling problem: some approaches}
\label{subsection:ApprITC2007}

In this section the significant techniques applied to the ITC 2007 - Examination timetabling track are described. This timetabling problem comprises 12 instances of different degree of complexity. Through the available website, competitors could submit their solutions for the given benchmark instances. Submitted solutions were evaluated as follows. First, it is checked if the solution is feasible and a so-called distance to the feasibility is computed. If it is feasible, the solution is further evaluated based on the fitness function, which measures the soft constraints total penalty. Then, competitors' solutions are ranked based on the distance to feasibility and solution's fitness value. The competitor with lower distance to feasibility value is the winner. In the case of a tie, the competitor's solution with the lowest fitness value wins. A solution is considered feasible if the value of distance to feasibility is zero. The set of hard constraints is the following:
\begin{itemize}
	\item no student must be elected to be present in more than one exam at the same time;
	\item the number of students in a class must not exceed the room's limit capacity;
	\item exam's length must not surpass the length of the assigned timeslot;
	\item exams ordering hard constraints must be followed; e.g., $Exam_1$ must be scheduled after $Exam_2$;
	\item room assignments hard constraints must be followed; e.g., 	$Exam_1$ must be scheduled in $Room_1$.
\end{itemize}

It is also necessary to compute the fitness value of the solution and so consider the soft constraints that were not obeyed. The soft constraints are listed below:
\begin{itemize}
	\item two exams in a row - a student should not be assigned to be in two adjacent exams in the same day;
	\item two exams in a day - a student should not be assigned to be in two non adjacent exams in the same day;
	\item period spread - reduce the number of times a student is assigned to be in two exams that are \textit{N} timeslots apart;
	\item mixed durations - reduce the number of exams with different durations that occur in a room and period;
	\item larger exams constraints - reduce the number of large exams that occur later in the timetable;
	\item room penalty - avoid assigning exams to rooms with penalty;
	\item period penalty - avoid assigning exams to periods with penalty.
\end{itemize}

To get a detailed explanation on how to compute the values of fitness and distance to feasibility based on the weight of each constraint, please check ITC 2007's website~\cite{McCollum2008}\\

%The finalists are ranked based their rankings on the instances. For full details on the rankings system, please consult~\cite{BarryMcCollum2008}.\\

In this thesis, we review some of the winners approaches. The winners list of the ITC 2007 competition is as follows:
\begin{itemize}
	\item 1st Place: Tom\'{a}\v{s} M\"{u}ller
	\item 2nd Place: Christos Gogos
	\item 3rd Place: Mitsunori Atsuta, Koji Nonobe, and Toshihide Ibaraki
	\item 4th Place: Geoffrey De Smet
	\item 5th Place: Nelishia Pillay
\end{itemize}

\paragraph{Tom\'{a}\v{s} M\"{u}ller's approach}

Tom\'{a}\v{s} M\"{u}ller's approach~\cite{Mueller2009} was actually used to solve all three problems established by the ITC 2007 competition. He was able to win two of them and be finalist on the third. For solving the problems, he opted for an hybrid approach, organized in a two-phase algorithm. In the first phase, Tom\'{a}\v{s} used Iterative Forward Search (IFS) algorithm~\cite{Mueller2005} to obtain feasible solutions and Conflict-based Statistics~\cite{Mueller2004} to prevent IFS from looping. 
	%--The timetabling problems solved were specified as constraint satisfaction problems, where events (exams, courses) are represented by variables. The variables' values are the possible pairs (time slot, room) that don't cause violations in the hard constraints. For each iteration there's an attempt to sign a value to an unassigned variable. If it violates hard constraints, the conflicting variables are unassigned. The variable chosen in each iteration is randomized or parameterized to, for example, assign the most difficult assignable exam first. The Conflict-based Statistics was used to memorize some previously passed conflicts and avoid repeating those for each iteration.\\
The second phase consists in using multiple optimization algorithms. These algorithms are applied using this order: \gls{hc} \cite{Russell2010}, \gls{gd} \cite{Dueck1993} and optionally \gls{sa}~\cite{Kirkpatrick1983}.\\

	%--\gls{hc} is used to optimize the first phase solution, resulting on a solution stuck at a local optimum. To leave local optimum area, \gls{gd} is used in order to try and result in a better solution. For last, \gls{sa} is used in a loop, keeping the temperature limit unchanged. After not getting a better solution for a limited time, the temperature is reheated (temperature limit gets higher) and \gls{hc} phase is used again forming a loop in these 3 optimization algorithms.\\

\paragraph{Christos Gogos' approach}

Gogos was able to reach second place in Examination Timetabling track, right after Muller. Gogos' approach~\cite{Gogos2012}, like M\"{u}ller's, was divided in two phases named \textit{Construction Phase} and \textit{Improvement Phase}, in which the first is used to construct a feasible solution and the second improves that solution, using local search algorithms. In the first phase, is starts using a Pre-processing stage, dealing with hidden dependencies between exams, prior to searching for a solution. 
	%--This stage deals with hidden dependencies, which means it adds dependencies to the problem that weren't there at the beginning but they exist and makes sense in the problem itself. For example, one hidden dependency may be one exam $E_1$ having an ordering constraint with exam $E_2$ stating that $E_2$ must be scheduled after exam $E_1$ and exam $E_3$ must be scheduled after exam $E_2$. So a dependency/constraint must be added so exam $E_3$ must be scheduled after exam $E_1$ as well. The key idea is that this pre-process method will be very helpful in the further stages when trying to find solutions.\\
After the pre-processing stage, a construction stage takes place, resulting in the best feasible solution given a limited period of time. 
	%--This method is used to create a complete timetable. In this stage, multiple solutions are attempted considering the available time in which only the best solution is passed to the next phase. The solution is constructed by constantly setting an exam and a room to a period until a feasible solution is found.\\
In the second phase, optimization methods are applied in this order: \gls{hc}, \gls{sa}, \gls{ip} that uses Branch and Bound, finishing with the Shaking Stage. Shaking Stage \textit{shakes} the current solution creating an equally good solution that is given to the \gls{hc}, creating a loop on the used optimization methods.\\
	%--\gls{hc} is used until reaching a local optimum. This algorithm is used to generate neighbor solutions, accepting only better solutions compared to the current one, until reaching a local optimum. It is considered local optimum when \gls{hc} cannot generate a better solution in a number of tries or time limit.\\

	%--The next stage consists in utilizing \gls{sa} to get out of local optimum. Simulated Annealing stops when it cannot generate better solutions after the specified period of time.

	%--After \gls{sa}, an \gls{ip} formulation that uses Branch and Bound (BB) is used in a set of sub-problems. This stage is called "\gls{ip} Sub-Problems Stage". This simply examines all periods trying to discover some possible improvements that can be done, for instance, swapping a room with high penalty with a room with no penalty or having lower penalty that is not currently being used.\\

	%--The last stage is the Shaking Stage. This method "shakes" the current best solution in order to get equally good solutions and pass it to \gls{sa} stage. This method is not used if the problem we're solving only has one room or the cost of each room is zero. Two heuristics are used in this stage. The first creates neighbor solutions but a solution is only accepted if the new solution is equally good and the room arrangement is better (total cost of room assignment is lower) as compared to the current best solution. The second heuristic reschedule a set of exams presented on the previous solution. These are first removed from the timetable and then scheduled using the same techniques used in the construction stage. This heuristic will probably generate a worse solution but it is accepted and used next on the \gls{sa} stage.\\

	%--Tabu Search (TS)~\cite{Talbi2009} is also used in this approach, even though it's not used as a standalone method. This one is used along with \gls{hc} and \gls{sa} algorithms to prevent them from looping and creating the same results. TS avoids the creation of equal neighbor solutions by not letting the same period be selected for swapping after this being selected once. This period can only be selected again after a certain number of swaps on other periods.\\

\paragraph{Mitsunori Atsuta, Koji Nonobe and Toshihide Ibaraki's approach}
Mitsunori Atsuta, Koji Nonobe and Toshihide Ibaraki ended up in third place on the Examination Timetabling track and won third and second place on the other tracks as well, utilizing the same approach for all of them. The approach~\cite{Atsuta2007} consists on applying constraint satisfaction problem solver adopting an hybridization of \gls{ts} and Iterated Local Search (modification of local search method), handling weight constraints {\color{red} ?}.\\

\paragraph{Geoffrey De Smet's approach}
Geoffrey De Smet's approach \cite{Smet2007} differs from all others because he decided not to use a known problem-specific heuristic to obtain a feasible solution, but instead used what is called the \textit{Drool's rule engine}, named \textit{drools-solver} \cite{Drools}. By definition, the drools-solver is a combination of optimization heuristics and meta-heuristics with very efficient score calculation. After obtaining a feasible solution, Geoffrey opted to use a local search algorithm to improve the solutions obtained using the drools-solver.\\

\paragraph{Nelishia Pillay's approach}
Nelishia Pillay opted to use a two-phase algorithm variation, using \textit{Developmental Approach based on Cell Biology} \cite{Pillay2007}, which the goal is to form a well-developed organism by the process of creating a cell, proceeding with cell division, cell interaction and cell migration. In this approach, each cell represents a timeslot. The first phase represents the creating of the first cell, cell division and cell interaction, and the second phase represents the cell migration.

%\subsection{Other approaches}
%\label{subsection:OtherAppr}

%\paragraph{(2009) Abdullah et al.'s approach}
%Salwani Abdullah, Hamza Turabieh, and Barry McCollum's approach \cite{Abdullah2009} consists on using an hybridization of an electromagnetic-like mechanism and the \gls{gd} algorithm. In this approach, the electromagnetism-like mechanism starts with a population of timetables randomly generated. Electromagnetic-like mechanism is a meta-heuristic algorithm using an \textit{attraction-repulsion} mechanism \cite{Javadian2008} to move the solutions to the optimum.\\

%\paragraph{(2009) Gogos et al.' approach}
%Christos Gogos, George Goulas, Panayiotis Alefragis and Efthymios Housos' approach \cite{Gogos2009} also used the two-phase algorithm. The first phase consists on creating the timetable using a greedy randomized positioning, with the use of a backtracking mechanism to help creating the solution. This construction phase is repeated and only the best solutions pass to the second phase. The second phase utilizes local search, \gls{sa}, shaking and \gls{ip} to improve the solutions. As mentioned by the author, this approach produced very good results  and can be compared to Tom\'{a}\v{s} M\"{u}ller's results in his approached used to win ITC 2007 competition.\\

%\paragraph{(2009) McCollum et al.'s approach}
%B. McCollum, P.J. McMullan, A. J. Parkes, E.K. Burke and S. Abdullah's two phased approach \cite{McCollum2009} use an adaptive ordering heuristic from \cite{Burke2004}, proceeding with an \textit{extended version} of \gls{gd}. As the author stated, this approach is robust and general considering the results obtained on the benchmark datasets from ITC 2007 using this approach.\\

%\paragraph{(2011) Malek Alzaqebah and Salwani Abdullah's approach}
%Malek Alzaqebah and Salwani Abdullah's two phased approach \cite{Alzaqebah2011} starts by using a graph coloring heuristic (largest degree ordering) to generate the initial solution and ends by utilizing the \textit{Artificial Bee Colony} search algorithm to optimize the solution.\\

%\paragraph{(2011) Hamza Turabieh and Salwani Abdullah's approach}
%Hamza Turabieh and SalwaniAbdullah's approach \cite{Turabieh2011} utilize two phase algorithm. The first phase consists on constructing initial solutions by using an hybridization of graph coloring heuristics (least saturation degree, largest degree first and largest enrollment {\color{red} ?} first). The second phase utilizes an hybridization of electromagnetic-like mechanism and \gls{gd} algorithm, just like in (2009) Abdullah et al.'s approach.\\

%\paragraph{(2012) Hamza Turabieh and Salwani Abdullah's approach}
%Hamza Turabieh and Salwani Abdullah created another approach in 2012 \cite{Abdullah2012}. It utilizes a Tabu-based memetic algorithm which consists on an hybridization of a genetic algorithm with a Tabu Search algorithm. The author states that this approach produces some of the best results when tested on ITC 2007's datasets.\\

%\paragraph{(2012) Demeester et al.'s approach}
%P. Demeester, B. Bilgin, P. De Causmaecker and G. V. Berghe used an hyper-heuristic approach \cite{Demeester2012}. The heuristics utilized are 'improved or equal' {\color{red} (hill climbing?)}, \gls{sa}, \gls{gd} and an adapted version of the \textit{late acceptance} strategy \cite{Burke2008}. These heuristics are used on already-created initial solutions. Initial solutions are constructed using an algorithm which does not guarantee the feasibility of the solution.\\

%\paragraph{(2012) McCollum et al.'s approach}
%B. McCollum, P.J. McMullan, A. J. Parkes, E.K. Burke and S. Abdullah's 2012 approach \cite{McCollum2012} to ITC 2007 problem was based on \gls{ip} formulation. This approach though, was not capable of solving the competition instances.\\

%\paragraph{(2012) Sabar et al.'s hyper-heuristic of hybridizations approach}
%N. R. Sabar, M. Ayob, R. Qu and G. Kendall utilized a graphical coloring hyper-heuristic on its approach \cite{Sabar2012}. This hyper-heuristic is composed of four hybridizations of these four methods: last degree, saturation degree, largest colored degree and largest enrollment. This approach seems to compete with the winners' approaches from ITC 2007, considering the benchmark results.\\


%\paragraph{(2012) Sabar et al.'s honey-bee approach}
%N. R. Sabar, M. Ayob, G. Kendall, R. Qub's approach \cite{Sabar2012a} utilizes a two phase algorithm. Starts by using an hybridization of graph coloring heuristics to obtain feasible solutions and a variant of honey-bee algorithm for optimization. The hybridization is composed of least saturation degree, largest degree first, largest enrollment first applied in this order.\\

%\paragraph{(2013) Salwani Abdullah and Malek Alzaqebah's approach}
%Salwani Abdullah and Malek Alzaqebah's opted to create an hybridization approach \cite{Abdullah2013}, mixing the utilization of a modified bees algorithm with local search algorithms (i.e. \gls{sa} and late acceptance \gls{hc})\\

%\paragraph{(2014) Salwani Abdullah and Malek Alzaqebah's approach}
%Salwani Abdullah and Malek Alzaqebah in 2014 constructed an approach \cite{Alzaqebah2014} that utilizes an hybridization of a modified artificial bee colony with local search algorithm (i.e. late acceptance \gls{hc}).\\

%\paragraph{(2014) Burke et al.'s approach}
%Edmund K. Burke, Rong Qu and Amr Soghier's approach \cite{Burke2014} uses an hyper-heuristic with hybridization of low level heuristics (neighbor operations) to improve the solutions. The low level heuristics are \textit{move exam}, \textit{swap exam}, \textit{kempe chain move} and \textit{swap timeslot}. After applying this hyper-heuristic with the hybridizations, the hybridization with the best results will be tested with multiple exam ordering methods, applying another hyper-heuristic with hybridizations. The heuristics applied are \textit{largest degree}, \textit{largest weighted degree}, \textit{saturation degree}, \textit{largest penalty} and \textit{random ordering}.\\

%\paragraph{(2014) Ali Hmer and Malek Mouhoub's approach}
%Ali Hmer and Malek Mouhoub's approach \cite{Mouhoub2014} uses a multi-phase hybridization of meta-heuristics. Works like the two phase algorithm but includes a pre-processing phase before the construction phase. This pre-processing phase is divided in two phases: the propagation of ordering constraints and implicit constraints discovery. The construction phase utilizes a variant of \gls{ts}. The optimization phase uses hybridization of \gls{hc}, \gls{sa} and an extended version of \gls{gd} algorithm.\\

%\paragraph{(2014) Ryan Hamilton-Bryce et al.'s approach}
%Ryan Hamilton-Bryce, Paul McMullan and Barry McCollum opted to use a non-stochastic method on their approach \cite{Hamilton-Bryce2014} when choosing examinations in the neighborhood searching process on the optimization phase. Instead, it uses a technique to make an intelligent selection of examinations using information gathered in the construction phase. This approach is divided in 3 phases. The first phase uses a \textit{Squeaky Wheel} constructor which generates multiple initial timetables and a weighted list for each timetable generated. Only the best timetable and its weighted list is passed to the second phase. The second phase is the \textit{Directed Selection Optimization} phase which uses the weighted list created in the construction phase to influence the selection of examinations for the neighbor search process. {\color{red} ("DSO utilizes the weighted list to influence the selection of the examination for optimization".. ??? Não deveria ser seleção de exames para criar soluções vizinhas ao qual estas vão para optimização? Ou a melhor vai para optimização..)}. Only the best timetable is passed onto the next phase. The third phase is the \textit{Highest Soft Constraint Optimization} phase which is similar to the previous phase but weighted list values are calculated based on the solution's individual soft constraints penalty.\\

%\paragraph{(2014) Syariza Rahman et al.'s constructive approach}
%S. A. Rahman, E. K. Burke, A. Bargiela, B. McCollum and E. Özcan's approach \cite{Rahman2014} is a constructive approach. This divides examinations in sets called \textit{easy sets} and \textit{hard sets}. Easy sets contain the examinations that are easy to schedule on a timetable and on the contrary, hard sets contain the ones that are hard to schedule and so are identified as the ones creating infeasibility. This allows to use the examinations present on the hard sets first on future construction attempts. There's also a sub-set within the easy set, called \textit{Boundary Set} which helps on the examinations' ordering and shuffling. Initial examinations' ordering  are accomplished by using graph coloring heuristics like largest degree and saturation degree heuristics.\\

%\paragraph{(2014) Syariza Rahman et al.'s adaptive linear combination approach}
%Syariza A. Rahman, Andrzej Bargiela, Edmund K. Burke, Ender Özcan, Barry McCollum and Paul McMullan's approach \cite{Rahman2014a} utilizes adaptive linear combinations of graph coloring heuristics (largest degree and saturation degree) with heuristic modifier. These adaptive linear combinations allows the attribution of difficulty scores to examinations depending on how hard their scheduling is. The ones with higher score, and so harder to schedule, are scheduled using two strategies: using single or multiple heuristics and with or without heuristic modifier. The authors conclude that multiple heuristics with heuristic modifier offers good quality solutions, and the presented adaptive linear combination is a highly effective method.\\

%\paragraph{(2014) Nuno Leite et al.'s approach}
%Nuno Leite, Fernando Melício and Agostinho C. Rosa's approach \cite{Leite2014} tries to solve not only the well-studied single epoch problem, but also an extension to that problem with two examination epochs. This approach utilizes a Memetic Algorithm to solve the problem. Memetic algorithms are hybridizations of a population-based meta-heuristic and a single-solution based meta-heuristic. The Memetic algorithm used is the hybridization of the \textit{Shuffled Complex Evolution Algorithm} and the Great Deluge algorithm. Shuffled Complex Evolution Algorithm is a Evolutionary Algorithm. Great Deluge algorithm is a variant of the local search \gls{sa}.\\