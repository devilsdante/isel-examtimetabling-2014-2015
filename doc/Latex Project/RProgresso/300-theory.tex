\chapter{State of the Art}
\label{stateofart}
\thispagestyle{plain}

In this section, we review the state-of-the-art of the problem at hand. We start by describing why timetabling is a rather complex problem, some possible approaches to solve it and some of the solutions already taken, specifically for the ITC 2007 benchmarks.\\

\section{Timetabling Problem}

When solving timetabling problems, it is possible to generate one of multiple types of solutions which are \textit{feasible}, \textit{non feasible}, \textit{optimal} or \textit{sub-optimal}. A feasible solution solves all the mandatory constraints, unlike non feasible solutions. An optimal solution is the best possible feasible solution given the problem constraints. A problem may have multiple optimal solutions. Lastly, non-optimal solutions are feasible solutions that have sub-optimal values.\\
\\
Timetabling automation is a subject that has been a target of research for about 50 years. The timetabling problem may be formulated as a search or an optimization problem~\cite{Schaerf1999}. As a search problem, the goal consists on finding a feasible solution that satisfies all the hard constraints, while ignoring the soft constraints. By posing the timetabling problem as an optimization problem, one seeks to minimize (considering a minimization problem) the violations of soft constraints while satisfying the hard constraints. Typically, the optimization is done after the use of a search procedure for finding an initial feasible solution.\\
\\
The basic examination timetabling problem, where only the \textit{clash} hard constraint is observed, reduces to the graph coloring problem~\cite{Jensen2001}, which is a well studied problem. The clash hard constraint specifies that no conflicting exams should be scheduled at the same time slot. Deciding whether a solution exists in the graph coloring problem, is a NP-complete problem~\cite{Arora2009}. Considering the graph coloring as an optimization problem, it is proven that the task of finding the optimal solution is also a NP-Hard problem \cite{Arora2009}. Graph Coloring problems are explained further in Section \ref{section:existingappr}
\\
%%%%%%%%%%%%%%%%%%%%
\section{Existing Approaches}
\label{section:existingappr}
%%%%%%%%%%%%%%%%%%%%
\begin{figure}[h!]
 \centering
   \includegraphics{./images/typesOfAlgorithms}
   \caption{Optimization methods (adapted from \cite{Talbi2009}).}
   \label{fig:TypesAlgorithms}
\end{figure}

Figure~\ref{fig:TypesAlgorithms} depicts a taxonomy for the known optimization methods. These methods are divided into \textit{Exact methods} and \textit{Approximate methods}.\\
\\
Timetabling solution approaches are usually divided into the following categories~\cite{Qu2009}: \textit{exact algorithms} (Branch-and-Bound, Dynamic Programming), \textit{graph based sequential techniques}, \textit{local search based techniques} (Tabu Search, Simulated Annealing, Great Deluge), \textit{population based algorithms} (Evolutionary Algorithms, Memetic algorithms, Ant Colony algorithms, Artificial immune algorithms), \textit{Multi-criteria techniques}, \textit{Hyper-heuristics}, \textit{Decomposition/clustering techniques}. Hybrid algorithms, which combine features of several algorithms, comprise the state-of-the-art. Due to its complexity, approaching the examination timetabling problem using exact method approaches can only be done for small size instances. Real problem instances found in practice are usually of large size, making the use of exact methods impracticable. Heuristic solution algorithms have been usually employed to solve this problem.\\
\\
Real problem instances are usually solved by applying algorithms which use both \textit{heuristics} and \textit{meta-heuristics}. Heuristic algorithms are problem-dependent, meaning that these are adapted to a specific problem in in which one can take advantage of its details. Heuristics are used to generate a feasible solution, focusing on solving all hard constraints only. Meta-heuristics, on the other hand, are problem-independent. These are used to, given the feasible solution obtained using heuristic algorithms, generate a better solution focusing on satisfying as many soft constraints as possible.\\
\\
Most of the existing meta-heuristic algorithms belong to one of the following three categories: One-Stage algorithms, Two-Stage algorithms and algorithms that allow relaxations~\cite{Lewis2007}. The One-Stage algorithm is used to get a solution, which the goal is to satisfy both hard and soft constraints at the same time. The Two-Stage algorithms are the most used types of approaches. This category is divided in two phases: the first phase consists in not considering the soft constraints and focusing on solving hard constraints to obtain a feasible solution; the second phase is an attempt to find the best solution, trying to solve the largest number of soft constraints as possible, given the solution of the first phase. Algorithms that allow relaxation are algorithms that can weaken some constraints in order to solve the \textit{relaxed problem} while considering the elimination the used relaxations.

\subsection{Exact methods}
\label{subsection:exactmethods}
Approximation algorithms like heuristics and meta-heuristics proceed to enumerate partially the search space and, for that reason, they can't guarantee finding the optimal solution. Exact approaches perform a complete enumeration of the search space and thus guarantee that the encountered solution is optimal. A negative aspect is the time taken to find the solution. If the decision problem is very difficult (e.g. NP-Complete), in practical scenarios, given large size problem instances, using this approach may not be possible due to the long execution time.\\

\subsubsection{Constraint-Programming Based Technique}
The Constraint Programming Based Technique (CPBT) allows direct programming with constraints which gives ease and flexibility in solving problems like timetabling. Two important features about this technique are the use of backtracking and logical variables. Constraint programming is different from other types of programming, as in these types it is specified the steps that need to be executed, but in constraint programming only the properties (hard constraints) of the solution, or the properties that should not be in the solution, are specified \cite{Qu2009}.\\

\subsubsection{Integer Programming}
The \gls{ip} is a mathematical programming technique in which the optimization problem to be solved must be formulated as an Integer Problem. If the objective function and the constraints must be linear, and all problem variables are integer valued, then the \gls{ip} problem is termed \gls{ilp}. If there are some variables that are continuous and other are integer, then the problem is called \gls{milp}. Schaerf~\cite{Schaerf1999} surveys some approaches using the \gls{milp} technique to school, course, and examination timetabling.

\subsection{Graph Coloring Based Techniques}
\label{subsection:graphcoloring}
As mentioned previously, timetabling problems can be reduced to a graph coloring problem. Considering this relation between the two problems, several authors used two-phased algorithms in which graph coloring heuristics were applied in the first phase, to obtain an initial feasible solution.\\

\subsubsection{Graph Coloring Problem}
The \gls{gc} problem consists in assigning colors to an element type of a graph which must follow certain constraints. The simplest sub-type is the \textit{vertex coloring}, which the main goal is to, given a number of vertices and edges, color the vertices so that no adjacent vertices have the same color. In this case the goal is to find a solution with the lowest number of colors as possible.\\
\\
The examination timetabling problem can be transformed into a graph coloring problem in the following way. The exams corresponds to vertices and there exists an edge connecting each pair of conflicting exams (exams that have students in common). With this mapping only the clash hard constraint is take into consideration. Thus, soft constraints are ignored. \cite{Qu2009}.\\
\\
Given the mapping between the \gls{gc} problem and the examination timetabling problem, \gls{gc} heuristics like \textit{Saturation Degree Ordering} are very commonly used to get the initial solutions. Others like \textit{First Fit} and other \textit{Degree Based Ordering} techniques (\textit{Largest Degree Ordering}, \textit{Incidence Degree Ordering}) are also heuristic techniques for coloring graphs~\cite{Carter1996}.\\

%\paragraph{Saturation Degree Ordering}
%The Saturation Degree Ordering heuristic colors the vertices with more constraints first. The coloring method is as follows: while choosing a vertice to color, the ones with higher saturation degree will be colored first. The saturation degree of one vertice is the number of differently colored vertices adjacent to this vertice or, in another words, the number of different colors of all adjacent vertices. In the case of a tie, the highest saturation vertice with higher number of adjacent vertices is chosen.

\subsection{Meta-heuristics}
Meta-heuristics, as mentioned above, usually provide solutions for optimization problems. In timetabling problems, meta-heuristic algorithms are used to optimize the feasible solutions provided by heuristics, like the \gls{gc} heuristics. Meta-heuristics are divided in two main sub-types, which are \textit{Single-solution meta-heuristics} and \textit{Population-based meta-heuristics}~\cite{Talbi2009}.\\

\subsubsection{Single-solution meta-heuristics}
Single-solution meta-heuristics' main goal is to modify and to optimize one single solution, maintaining the search focused on local regions. This type of meta-heuristic is therefore exploitation oriented. Some examples of this type are \textit{\gls{sa}}, \textit{Variable-Neighborhood Search}, \textit{\gls{ts}}, and \textit{Guided Local Search}~\cite{Talbi2009}. \\

\subsubsection{Population-based meta-heuristics}
Population-based meta-heuristics' main goal is to modify and to optimize multiple candidate solutions, maintaining the search focused in the whole space. This type of meta-heuristic is therefore exploration oriented. Some examples of this type are \textit{Particle Swarm}, \textit{Evolutionary Algorithms}, and \textit{Genetic Algorithms}~\cite{Talbi2009}.\\


\subsection{ITC 2007 Examination timetabling problem: some approaches}
\label{subsection:ApprITC2007}

In this section, the five ITC 2007 - Examination timetabling track - finalists' approaches are described. This timetabling problem comprises 12 instances of different degree of complexity. Through the available website, competitors could submit their solutions for the given benchmark instances. Submitted solutions were evaluated as follows. First, it is checked if the solution is feasible and a so-called distance to the feasibility is computed. If it is feasible, the solution is further evaluated based on the fitness function, which measures the soft constraints total penalty. Then, competitors' solutions are ranked based on the distance to feasibility and solution's fitness value. The competitor's method achieving the lower distance to feasibility value is the winner. In the case of a tie, the competitor's solution with the lowest fitness value wins. A solution is considered feasible if the value of distance to feasibility is zero. The set of hard constraints is the following:
\begin{itemize}
	\item no student must be elected to be present in more than one exam at the same time;
	\item the number of students in a class must not exceed the room's capacity;
	\item exam's length must not surpass the length of the assigned timeslot;
	\item exams ordering hard constraints must be followed; e.g., $Exam_1$ must be scheduled after $Exam_2$;
	\item room assignments hard constraints must be followed; e.g., 	$Exam_1$ must be scheduled in $Room_1$.
\end{itemize}

It is also necessary to compute the fitness value of the solution which is calculated as an average sum of the soft constraints penalty. The soft constraints are listed below:
\begin{itemize}
	\item two exams in a row - a student should not be assigned to be in two adjacent exams in the same day;
	\item two exams in a day - a student should not be assigned to be in two non adjacent exams in the same day;
	\item period spread - reduce the number of times a student is assigned to be in two exams that are \textit{N} timeslots apart;
	\item mixed durations - reduce the number of exams with different durations that occur in the same room and period;
	\item larger exams constraints - reduce the number of large exams that occur later in the timetable;
	\item room penalty - avoid assigning exams to rooms with penalty;
	\item period penalty - avoid assigning exams to periods with penalty.
\end{itemize}

To get a detailed explanation on how to compute the values of fitness and distance to feasibility based on the weight of each constraint, please check the ITC 2007 website~\cite{McCollum2008}.\\

%The finalists are ranked based their rankings on the instances. For full details on the rankings system, please consult~\cite{BarryMcCollum2008}.\\

Next we review the ITC 2007's five winners approaches. The winners list of the ITC 2007 competition is as follows:
\begin{itemize}
	\item 1st Place - Tom\'{a}\v{s} M\"{u}ller
	\item 2nd Place - Christos Gogos
	\item 3rd Place - Mitsunori Atsuta, Koji Nonobe, and Toshihide Ibaraki
	\item 4th Place - Geoffrey De Smet
	\item 5th Place - Nelishia Pillay
\end{itemize}

We now briefly describe these approaches.

Tom\'{a}\v{s} M\"{u}ller's approach~\cite{Mueller2009} was actually used to solve all three problems established by the ITC 2007 competition. He was able to win two of them and to be finalist on the third. For solving the problems, he opted for an hybrid approach, organized in a two-phase algorithm. In the first phase, Tom\'{a}\v{s} used the Iterative Forward Search (IFS) algorithm~\cite{Mueller2005} to obtain feasible solutions and Conflict-based Statistics~\cite{Mueller2004} to prevent IFS from looping. The second phase consists in using multiple optimization algorithms. These algorithms are applied using this order: \gls{hc} \cite{Russell2010}, \gls{gd} \cite{Dueck1993}, and optionally \gls{sa}~\cite{Kirkpatrick1983}.\\

	%--\gls{hc} is used to optimize the first phase solution, resulting on a solution stuck at a local optimum. To leave local optimum area, \gls{gd} is used in order to try and result in a better solution. For last, \gls{sa} is used in a loop, keeping the temperature limit unchanged. After not getting a better solution for a limited time, the temperature is reheated (temperature limit gets higher) and \gls{hc} phase is used again forming a loop in these 3 optimization algorithms.\\

Gogos was able to reach second place in Examination Timetabling track, right after M\"{u}ller. Gogos' approach~\cite{Gogos2012}, like M\"{u}ller's, is a two-phase approach but with a pre-processing stage. The first phase starts by the application of a Pre-processing stage, in which the hidden dependencies between exams are discovered in order to speed up the optimization phase. After the pre-processing stage, a construction stage takes place, using a meta-heuristic called \textit{\gls{grasp}}. In the second phase, optimization methods are applied in this order: \gls{hc}, \gls{sa}, \gls{ip} (the Branch and Bound procedure), finishing with the so-called Shaking Stage. Shaking Stage \textit{shakes} the current solution creating an equally good solution that is given to the \gls{hc}, creating a loop on the used optimization methods.\\

Mitsunori Atsuta, Koji Nonobe and Toshihide Ibaraki ended up in third place on the Examination Timetabling track and won third and second place on the other tracks as well, with the same approach for all of them. The approach~\cite{Atsuta2007} consists on applying a constraint satisfaction problem solver adopting an hybridization of \gls{ts} and Iterated Local Search.\\

Geoffrey De Smet's approach \cite{Smet2007} differs from all others because he decided not to use a known problem-specific heuristic to obtain a feasible solution, but instead used what is called the \textit{Drool's rule engine}, named \textit{drools-solver} \cite{Drools}. By definition, the drools-solver is a combination of optimization heuristics and meta-heuristics with very efficient score calculation. A solution's score is the sum of the weight of the constraints being broken. After obtaining a feasible solution, Geoffrey opted to use a local search algorithm to improve the solutions obtained using the drools-solver.\\

Nelishia Pillay opted to use a two-phase algorithm variation, using \textit{Developmental Approach based on Cell Biology} \cite{Pillay2007}, which the goal consists in forming a well-developed organism by the process of creating a cell, proceeding with cell division, cell interaction and cell migration. In this approach, each cell represents a timeslot. The first phase represents the process of creating the first cell, cell division and cell interaction. The second phase represents the cell migration.