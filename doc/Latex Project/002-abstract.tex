\chapter*{Acknowledgments}
\addcontentsline{toc}{chapter}{Acknowledgments} 

I would like to express my gratitude to my supervisors Artur Ferreira and Nuno Leite, during the entire project development. If it wasn't for them, the development of this project wouldn't be possible. Their assistance and motivation really made me to continue and to work hard on this project.\\
\\
I would like to thank my best friends João Vaz, Daniel Albuquerque, André Ferreira, Pedro Miguel, my brothers Pedro Nunes and João Tiago, and my parents, Mariana and Paulo, for always being there to support me on the hardest moments of my life. Their help on my personal life issues were crucial while developing this project.\\
\\
Last, but not the least, I would like to thank Instituto Superior de Engenharia de Lisboa, for giving me support and knowledge in the Computing Engineering subject, for the past six years.
\
\chapter*{Resumo}

\addcontentsline{toc}{chapter}{Resumo} 
Nos últimos anos, o tema geração automática de horários tem sido alvo de muito estudo. Em muitas instituições, a elaboração de horários ainda é feita manualmente, constituindo-se uma tarefa demorada e penosa para instâncias de grande dimensão. Outro problema recorrente na abordagem manual é a existência de falhas dada a dificuldade do processo de verificação, e também a qualidade final do horário produzido. Se este fosse criado por computador, o horário seria válido e seriam de esperar horários com qualidade superior dada a capacidade do computador para pesquisar o espaço de soluções.\\
\\
A elaboração de horários não é uma tarefa fácil, mesmo para uma máquina. Por exemplo, horários escolares necessitam de seguir certas regras para que seja possível a criação de um horário válido. Mas como o espaço de estados (soluções) válidas é tão vasto, é altamente improvável criar um algoritmo que faça a enumeração completa de soluções a fim de escolher a melhor solução possível, considerando as constituições do problema. A utilização de algoritmos que realizam a enumeração implícita de soluções (por exemplo, branch and bound), não é viável para problemas de grande dimensão. Daí a utilização de heurísticas que percorrem de uma forma guiada o espaço de estados, conseguindo assim uma solução razoável em tempo útil.\\
\\
Um dos objetivos do projeto consiste na criação duma abordagem que siga as regras do \textit{International Timetabling Competition (ITC) 2007} incidindo na criação de horários de exames em universidades (Examination timetabling track). Este projeto utiliza uma abordagem de heurísticas híbridas. Isto significa que utiliza múltiplas heurísticas para obter a melhor solução possível. Utiliza uma variação da heurística de Graph Coloring para obter uma solução válida e as meta-heurísticas Simulated Annealing e Hill Climbing para melhorar a solução obtida.\\
\\
Os resultados finais são satisfatórios, pois em algumas instâncias os resultados são melhores do que alguns dos cinco finalistas do concurso ITC 2007.
\
\section*{Palavras Chave}

Heurísticas; Meta-heurísticas; International Timetabling Competition 2007; Horários; Horários de exames; Graph Coloring; Simulated Annealing;
\

\chapter*{Abstract}

\addcontentsline{toc}{chapter}{Abstract} 
In the last few years the automatic creation of timetables is being a well-studied subject. In many institutions, the elaboration of timetables is still manual, thus being a time-consuming and difficulty task for large instances. Another current problem in the manual approach is the existence of failures given the difficulty in the process verification, and so the quality of the produced timetable. If this was created by a computer, the timetable would be valid and would expect timetables with better quality given the computer's capacity to search the solution space. \\
\\
It is not easy to elaborate timetables, even for a machine. For example, scholar/university timetables need to follow certain type of constraints or rules for them to be considered valid. But since the solution space is so vast, it is highly unlikely to create an algorithm that completely enumerates the solutions in order to choose the best solution possible, considering the problem constitutions. The use of algorithms that perform implicit enumeration solutions (for example, an branch bound), is not feasible for large problems. Hence the use of heuristics which navigate through the solution space in a guided way, obtaining then a reasonable solution in acceptable time.\\
\\
One main objective of this project consists in creating an approach that follows the \textit{International Timetabling Competition (ITC) 2007} rules, focusing on creating examination timetables. This project will use a hybrid approach. This means it will use an approach that includes multiple heuristics in order to find the best possible solution. This approach uses a variant of the Graph Coloring heuristic to find an initial valid solution, and the meta-heuristics Simulated Annealing and Hill Climbing to improve that solution.\\
\\
The final results are satisfactory, as in some instances the obtained results match the results of the five finalists from ITC 2007.


\section*{Keywords}

Heuristics; Meta-heuristics; International Timetabling Competition 2007; Timetable; Examination Timetabling; Graph Coloring; Simulated Annealing;

