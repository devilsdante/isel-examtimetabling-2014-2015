\chapter{\textcolor{green}{Conclusions and Future Work}}
\label{chap:FutureWork}

In this project, an algorithm was developed to solve the timetabling problem. It was the main objective to use the problem described in the \gls{itc2007}, and so the instances presented in this competition were used. The developed algorithm utilizes hybrid meta-heuristics, named \gls{gc}, \gls{sa}, and \gls{hc}, to try and solve the problem.\\
\\
With the use of this hybrid approach of \gls{gc}, \gls{sa}, and \gls{hc} we've reached near the results of the five finalists of the \gls{itc2007} on most of the datasets. In some of the datasets, we were able to be right before M\"{u}ller's results.\\
\\
Our results were compared against the top five contestants of the competition, as well as with state-of-the-art approaches. These latter approaches were able to beat M\"{u}ller's results on most of the datasets, but M\"{u}ller's approach can still get very good results on all the datasets. The Muller's results contrast with some approaches that can beat their results on some datasets but get much worse results on others.\\
\\
We can conclude that the \gls{sa} works as predicted, getting much better results on most of the datasets when executed for much more time. Running the current approach for 12 hours, yielded better results, as compared to M\"{u}ller's and some recent approaches.\\
\\
There are different approaches to be tested with the implemented approach in order to try to improve the results. Some new approaches may include the use of more heuristics with the \gls{sa} and \gls{hc}, such as \gls{gd}. One improvement that must be implemented in order to get better results is editing the implemented \gls{sa} and make it able to move more than one examination at each time, to allow examinations with \textit{exam\_coincidence} hard constraint to be moved. Another possible improvement would be to test the \gls{sa} algorithm with different parameters and check if can it can yield better results.