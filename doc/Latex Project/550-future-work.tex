\chapter{\textcolor{green}{Conclusions and Future Work}}
\label{chap:FutureWork}

We were able to conclude that with the use of this hybrid approach of \gls{gc}, \gls{sa}, and \gls{hc} we matched the results of the five winners of the \gls{itc2007} on most of the sets. It should be possible to be able to get better results if more tests were to be performed on the parameters of the \gls{sa} meta-heuristic and upgrade the \gls{sa} to be able to move the examinations with \textit{exam\_coincidence} hard constraint.\\
\\
The tests were compared no only with the top five contestants of the competition, but also with more up-to-date approaches. Up-to-date approaches were able to beat M\"{u}ller's results on most of the sets, but M\"{u}ller's approach can still get very good results on all the sets, unlike some approaches that can beat his results on some sets and get much worse results on others.\\
\\
We can conclude that the \gls{sa} works as predicted, getting much better results on most of the sets when executed for much more time. Running the current approach for 12 hours, generated better results compared to M\"{u}ller's and some up-do-date approaches.\\
\\
There are unlimited different approaches to be tested with the implemented approach in order to try to improve the results. Some new approaches may include the use of more heuristics with the \gls{sa} and \gls{hc}, like the \gls{gd}. One upgrade that must be implemented in order to get better results is editing the implemented \gls{sa} and make it able to move more than one examination at the time to allow examinations with \textit{exam\_coincidence} hard constraint to be moved.