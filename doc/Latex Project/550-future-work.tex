\chapter{Future Work}
\label{chapter:FutureWork}

In the future work, we plan on getting better results on all the sets by tweaking some features in the implemented code. This includes changing the input parameters for each set to obtain the best performance in order to deliver the best results on each one of them.\\
\\
One of the main features that needs to be upgraded is the performance of the simulated annealing (and of course the hill climbing which in this case is part of the simulated annealing), because this heuristic for each new generated neighbor, it computes its fitness value again, from the beginning. As some tests were made, we concluded that most of the time spent on this heuristic is used by the computation of the fitness value for each neighbor. We should get better results if the fitness value was computed based on the change that was made on the new neighbor, instead of computing it all again. First we need to study if this feature is possible to implement. If it is, the performance boost should be very noticeable.\\
\\
The graph coloring heuristic will probably be edit in order to try and get a feasible solution on all the sets, since the 4th set probably ends up in an infinite loop, and so it gets unfeasible.\\
\\
Apart from the changes on the existing code, if there's time to implement new heuristics to test and compare to the top 5 winners, we plan on implementing a population-based algorithm like the Genetic Algorithm or a  to be used as optimization methods after the already implemented graph coloring.