\chapter{Implementation}
\label{implementation}
\thispagestyle{plain}

This project is meant to participate in the ITC2007. Comparing the approach and the results with the finalists of the competition is one of the main objectives of this project. Thus, it is ideal to develop an organized program with the best performance possible, so the implemented code can be easily understandable, extensible and so afford the best scores possible when creating timetable solutions. Taking that into account, this section is reserved to the description of the architecture and explanation of the most important decisions taken regarding this subject.

\section{System Arquitecture}

The architecture of this project is divided in multiple layers. These are independent from one another and each of them has its unique features considering the objectives of this project. The layers presented in the project are named \textit{Data Layer}, \textit{Data Access Layer}, \textit{Business Layer}, \textit{Heuristics Layer}, \textit{Tools Layer} and \textit{Presentation Layer}. \\

The assortment, dependencies and the main classes of each layer can be seen in Figure X.

\subsection{Data Layer}
\subsection{Data Access Layer}
\subsection{Business Layer}
\subsection{Heuristics Layer}
\subsection{Tools Layer}
\subsection{Presentation Layer}



\section{Loader Module}

- Analysis of benchmark data, constraints, ....

Class diagram


\section{Solution Method}

\subsection{Graph Coloring}

\subsection{Simulated Annealing}

SEE Lecture Notes

\subsection{Neighborhood Operators}

SEE Muller









